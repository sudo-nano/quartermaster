\documentclass{article}

\usepackage{hyperref, minted}

\title{Quartermaster Developer Documentation}
\author{sudo-nano}

\begin{document}
\maketitle

\tableofcontents

\section{Interface}
The primary interface of the \verb|quartermaster| program is the \verb|prompt()| loop in \verb|main.py|. It provides the user with a prompt, then takes their input and \verb|match|es it to the list of valid commands. If it matches one, then it runs the appropriate function from \verb|mechanics.py|. 


\section{Commands}

\subsection{calc}
Usage: \verb|calc <recipe> <quantity>| \\

The \verb|calc| command takes two parameters, a recipe and the quantity of that recipe. For example, 

\section{Data Formatting}

\subsection{DataSet Class}
The \verb|DataSet| class is an object type for holding all the imported data in a \verb|quartermaster| session. There is currently only one \verb|DataSet| object, called \verb|session|. 
All data imported from files is loaded into the session DataSet object. 

\subsection{TOML file formatting}
Each type of data is stored in separate TOML files. Each TOML file must have \verb|type = "<type_here>"| to indicate its type to the program. If it doesn't have this, the program will refuse to load it. \\

Currently implemented types: 
\begin{itemize}
	\item Ingredients
	\item Recipes
\end{itemize}

Types to implement later: 
\begin{itemize}
	\item People 
	\item Groups
	\item Meal plans 
	\item Sessions
\end{itemize}

\subsection{Ingredient Files}
Ingredient files must have \verb|type = "ingredients"| to indicate that the file holds only ingredients, otherwise they will not be imported. This is meant to prevent accidentally
importing other types into the session ingredient dataset. \\

Each ingredient is one table in its TOML file. Example below. See the included test\_ingredients.toml file for more examples. 

\begin{minted}{TOML}
[rice]                  # Ingredient name in header
unit = "gram"           # Unit of measurement
price_per_unit = 0.01   # Price per unit of measurement 

# List of lists, each sub-list describing quantity of purchase 
# and cost of purchasing that quantity. 
purchase_increments = [ [1000.0, 5.00] ]

# List of diet incompatibilities that conflict with this 
# ingredient. Leave list empty if there are none. 
diet_incompat = []
\end{minted}

\subsection{Recipe Files}



\section{Planned Features}
\begin{itemize}
	\item Meal Plan object
	
	\begin{itemize}
		\item Allows easy planning for a limited subset of meal options 
		\item Ability to check meal plan compatibility with a set of people
		\item Suggest necessary modifications for incompatibilities?
	\end{itemize}

	\item Person object 

	\begin{itemize}
		\item Allows specification of dietary restrictions 
		\item Contains a float describing how many standard servings they will consume per meal 
	\end{itemize}

	\item Group object 

	\begin{itemize}
		\item Allows grouping of Person objects 
		\item Later, there will be a command to calculate the required supplies for a given group, meal plan, and length of time 
			to supply the group. 
	\end{itemize}

	\item Session object 
	\begin{itemize}
		\item Allows saving and loading of session state 
	\end{itemize}
\end{itemize}

\end{document}