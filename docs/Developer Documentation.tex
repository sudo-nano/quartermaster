\documentclass{article}

\usepackage{hyperref}

\title{Quartermaster Developer Documentation}
\author{sudo-nano}

\begin{document}
\maketitle

\tableofcontents

\section{Interface}
The primary interface of the \verb|quartermaster| program is the \verb|prompt()| loop in \verb|main.py|. It provides the user with a prompt, then takes their input and \verb|match|es it to the list of valid commands. If it matches one, then it runs the appropriate function from \verb|mechanics.py|. 


\section{Commands}

\subsection{calc}
The \verb|calc| command takes two parameters, a recipe and the quantity of that recipe. 

\section{Data Formatting}

\subsection{DataSet Class}
The \verb|DataSet| class is an object type for holding all the imported data in a \verb|quartermaster| session. There is currently only one \verb|DataSet| object, called \verb|default_dataset|. 
All data imported from files is loaded into the session DataSet object. 

\section{Planned Features}
\begin{itemize}
	\item Meal Plan object
	
	\begin{itemize}
		\item Allows easy planning for a limited subset of meal options 
		\item Ability to check meal plan compatibility with a set of people
		\item Suggest necessary modifications for incompatibilities?
	\end{itemize}

	\item Person object 

	\begin{itemize}
		\item Allows specification of dietary restrictions 
		\item Contains a float describing how many standard servings they will consume per meal 
	\end{itemize}

	\item Group object 

	\begin{itemize}
		\item Allows grouping of Person objects 
		\item Later, there will be a command to calculate the required supplies for a given group, meal plan, and length of time 
			to supply the group. 
	\end{itemize}

\end{itemize}

\end{document}